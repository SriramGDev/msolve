\documentclass[a4paper,english,11pt]{scrartcl}

\usepackage{ifpdf}
\ifpdf
  \pdfoutput=1
\fi

%\usepackage[DIV=9]{typearea}
\KOMAoptions{DIV=12,abstract=true}

%%%%% PAQUETS

% Base
\usepackage[T1]{fontenc}
\usepackage[utf8]{inputenc}
\usepackage[english]{babel}
\usepackage{amsmath,amssymb,amsfonts,amsthm}
\usepackage[usenames,dvipsnames,svgnames,table,hyperef]{xcolor}
\usepackage{eurosym}

% Typographies
\usepackage{microtype}
\usepackage{mathtools}  % Pour mathrlap et compagnie
\usepackage{booktabs}   % De beaux tableaux
\usepackage{array}
\newcolumntype{R}[1]{>{\raggedleft\let\newline\\\arraybackslash\hspace{0pt}}p{#1}}

\usepackage[]{csquotes}
\usepackage{xspace}
%\usepackage{fontawesome}

% Graphiques
%\usepackage{tikz}
%\usepackage{tikz-cd}
%\usetikzlibrary{calc,intersections,angles,quotes}

\usepackage{float}

\usepackage[]{graphicx}

% Divers
\usepackage[]{url}
\usepackage[backgroundcolor=white,linecolor=red]{todonotes}

\usepackage[shortlabels,inline]{enumitem}
\setlist[description]{labelindent=!,labelsep=1em,font=\normalfont\bfseries}

\usepackage{bm}

\usepackage{libertine}  % NOARXIV
\usepackage[libertine,vvarbb]{newtxmath}  % NOARXIV
\setkomafont{sectioning}{\bfseries\rmfamily}
\setkomafont{title}{}

\usepackage[ruled]{caption}      % La légende des boites flottantes
\captionsetup{format=hang,justification=raggedright,font=small,width=.95\textwidth,labelfont=it,labelsep=period}
\usepackage[titles]{tocloft}
\usepackage{setspace}
\setstretch{1.07}
\setdisplayskipstretch{.6}


% Algorithms
\usepackage[]{algpseudocode}
\floatstyle{ruled}
\newfloat{algo}{tp}{lop}
\floatname{algo}{Algorithm}
\setcounter{topnumber}{1}     % Pas plus d'une boite flottante en haut d'une page.
\setcounter{bottomnumber}{1}  % idem mais en bas

% scale texttt
\usepackage[scaled=.75]{beramono}
% PL : Chez moi, \tt, \sf, \bf, etc ne fonctionnent pas.
\def\sf{}    % Il faudrait remplacer par \mathsf mais la syntaxe n'est pas la même
\def\tt{\ttfamily}

\definecolor{darkgoldenrod}{rgb}{0.72, 0.53, 0.04}
\definecolor{amber}{rgb}{1.0, 0.75, 0.0}
\definecolor{mred}{rgb}{.886,.239,.157}
\definecolor{darkblue}{RGB}{63,101,134}
% \usepackage[]{tocstyle}
% \usetocstyle{nopagecolumn}
% \settocstylefeature{pagenumberhook}{\normalfont\itshape\color{darkgoldenrod}}
% \setkomafont{sectionentrypagenumber}{\itshape\color{darkgoldenrod}}

\typearea{10}

\setkomafont{title}{\raggedright\sffamily\bfseries\color{darkblue}}
\setkomafont{title}{\raggedright\sffamily\bfseries\color{darkgoldenrod}}
\addtokomafont{subtitle}{\normalfont\sffamily\raggedright\Large\itshape}
\setkomafont{subject}{\raggedright\Large\sffamily\medls}
\addtokomafont{author}{\raggedright\sffamily\setlength{\tabcolsep}{0pt}}
\setkomafont{pageheadfoot}{\sffamily}
\setkomafont{pagenumber}{\sffamily}

\setkomafont{titlehead}{\sffamily}

%\addtokomafont{section}{\color{darkblue}}
\addtokomafont{section}{\color{darkgoldenrod}}
%\addtokomafont{subsection}{\color{darkblue}}
\addtokomafont{subsection}{\color{darkgoldenrod}}
%\addtokomafont{subsubsection}{\normalfont\itshape\color{darkblue}}
\addtokomafont{subsubsection}{\normalfont\itshape\color{darkgoldenrod}}
%\addtokomafont{paragraph}{\normalfont\itshape\color{darkblue}}
\addtokomafont{paragraph}{\normalfont\itshape\color{darkgoldenrod}}

\ifpdf
  \usepackage[
      linktocpage=true,
      hyperfootnotes=false,
      pdftex,                %
      bookmarks         = true,%     % Signets
      bookmarksnumbered = true,%     % Signets numerotes
      pdfpagemode       = None,%     % Signets/vignettes fermes a l'ouverture
      pdfstartview      = FitH,%     % La page prend toute la largeur
      pdfpagelayout     = SinglePage,% Vue par page
      colorlinks        = true,%     % Liens en couleur
      linkcolor= red, %    % couleur des liens internes
      citecolor         =blue,
      urlcolor          = magenta,%  % Couleur des liens externes
      pdfborder         = {0 0 0}%   % Style de bordure : ici, pas de bordure
      ]{hyperref}%                   % Utilisation de HyperTeX
\else
  \usepackage[
      linktocpage=true,
      hyperfootnotes=false,
      bookmarks         = true,%     % Signets
      bookmarksnumbered = true,%     % Signets numerotes
      pdfpagemode       = None,%     % Signets/vignettes fermes a l'ouverture
      pdfstartview      = FitH,%     % La page prend toute la largeur
      pdfpagelayout     = SinglePage,% Vue par page
      colorlinks        = true,%     % Liens en couleur
      linkcolor= red, %    % couleur des liens internes
      citecolor         =blue,
      urlcolor          = magenta,%  % Couleur des liens externes
      pdfborder         = {0 0 0}%   % Style de bordure : ici, pas de bordure
      ]{hyperref}%                   % Utilisation de HyperTeX
\fi
    
\hypersetup{
linkcolor=blue!60!black,
citecolor=red, %OrangeRed, %Emerald, %OliveGreen,
urlcolor=blue, %BlueViolet %NavyBlue
}
\hypersetup{colorlinks=true}
\usepackage{framed}
%\usepackage{bm}

\graphicspath{{../fwf_anr_2017/etc/}}
\usepackage{tcolorbox}
\tcbuselibrary{skins}

\usepackage{pgfgantt}
\usepackage{pgfplots}
\usetikzlibrary{positioning}
\usetikzlibrary{fit}
\usetikzlibrary{backgrounds}
\usetikzlibrary{calc}
%%SG\usetikzlibrary{shapes,fpu,arrows,arrows.meta,calc,matrix,shapes.geometric,decorations.pathmorphing,intersections,fit,external,backgrounds,spy}
%\usetikzlibrary{mindmap}
%\usetikzlibrary{decorations.text}

%\tikzset{box1/.style={draw=black, thick, rectangle,rounded corners, minimum height=2cm, minimum width=2cm}}
\tikzset{box1/.style={draw=black, thick, rectangle,rounded corners, fill=black!7!white}}
\tikzset{box2/.style={draw=black, thick, rectangle,rounded corners, fill=black!7!white, minimum height=3cm}}
\tikzset{annot/.style={draw=black,thick,rounded corners,anchor=base, fill=red!10!white}}

\newtcolorbox[auto counter]{challenge}[2][]{
colframe=white!40!black,coltext=white!25!black,colback=white
,fonttitle=\bfseries, title=Exercise~\thetcbcounter}

\usepackage{fancyhdr}

\allowdisplaybreaks[4]

\newtheorem{proposition}{Proposition}
\newtheorem{lemma}[proposition]{Lemma}
\newtheorem{theorem}[proposition]{Theorem}
\newtheorem{corollary}[proposition]{Corollary}
\newtheorem{definition}[proposition]{Definition}
\newtheorem{conjecture}[proposition]{Conjecture}
\newtheorem*{question}{Problem}


% \def\mbbone{\mathbf{1}} % ONLYARXIV
\def\mbbone{\mathbb{1}} % NOARXIV

\theoremstyle{definition}
\newtheorem{example}[proposition]{Example}

\theoremstyle{remark}
%\newtheorem{example}[proposition]{Example}
\newtheorem*{remark}{Remark}

\newcommand{\cmment}[1]{}
\newcommand{\mop}[1]{\operatorname{#1}}
\newcommand{\ud}{\mathrm{d}}
\newcommand{\st}{\ \middle|\ }
%\newcommand{\eqdef}{\smash{\ \stackrel{\text{def}}{=}\ }}
\newcommand{\eqdef}{\triangleq}
\newcommand{\abs}[1]{\left|#1\right|}
\newcommand{\bA}{{\mathbb{A}}}
\newcommand{\bE}{{\mathbb{E}}}
\newcommand{\bP}{{\mathbb{P}}}
\newcommand{\bN}{{\mathbb{N}}}
\newcommand{\bC}{{\mathbb{C}}}
\newcommand{\bQ}{{\mathbb{Q}}}
\newcommand{\bR}{{\mathbb{R}}}
\newcommand{\bF}{{\mathbb{F}}}
\newcommand{\bK}{{\mathbb{K}}}
\newcommand{\bS}{{\mathbb{S}}}
\newcommand{\bZ}{{\mathbb{Z}}}
\newcommand{\cL}{{\mathcal{L}}}
\newcommand{\cB}{{\mathcal{B}}}
\newcommand{\cD}{{\mathcal{D}}}
\newcommand{\cY}{{\mathcal{Y}}}
\newcommand{\cT}{{\mathcal{T}}}
\newcommand{\cS}{{\mathcal{S}}}
\newcommand{\cN}{{\mathcal{N}}}
\newcommand{\cO}{{\mathcal{O}}}
\newcommand{\cH}{{\mathcal{H}}}
\newcommand{\cU}{{\mathcal{U}}}

\newcommand{\vol}{\mop{vol}}

\newcommand{\Sqrt}{\mop{Sqrt}}
\newcommand{\Sin}{\mop{Sin}}
\newcommand{\Cos}{\mop{Cos}}

\newcommand{\NJ}{\mop{NJ}}

\newcommand{\Proj}{\ensuremath{\bP^n}}
\newcommand{\intx}[1]{\int_{\mathrlap{#1}}}
\newcommand{\muav}{\mu_{\textrm{av}}}

\def\geq{\geqslant}
\def\leq{\leqslant}

\makeatletter
\def\cqlt{c.ql.t\@ifnextchar{.}{}{.\xspace}}

%\bf and KOMA do not work in jcapco TeX distribution it complains that \bf is an old command if KOMA is active
\DeclareOldFontCommand{\rm}{\normalfont\rmfamily}{\mathrm}
\DeclareOldFontCommand{\sf}{\normalfont\sffamily}{\mathsf}
\DeclareOldFontCommand{\tt}{\normalfont\ttfamily}{\mathtt}
\DeclareOldFontCommand{\bf}{\normalfont\bfseries}{\mathbf}
\DeclareOldFontCommand{\it}{\normalfont\itshape}{\mathit}
\DeclareOldFontCommand{\sl}{\normalfont\slshape}{\@nomath\sl}
\DeclareOldFontCommand{\sc}{\normalfont\scshape}{\@nomath\sc}

\makeatother

%\usepackage[pdftex,hypertexnames=false,hidelinks]{hyperref}


\def\e{{\bm{e}}}
\def\s{{\bm{s}}}
\def\x{{\bm{x}}}
\def\y{{\bm{y}}}
\def\z{{\bm{z}}}

\def\X{{\bm{X}}}
\def\Y{{\bm{Y}}}

\def\Z{\mathbb{Z}}
\def\K{\mathbb{K}}
\def\N{\mathbb{N}}
\def\P{\mathbb{P}}
\def\Q{\mathbb{Q}}
\def\R{\mathbb{R}}
\def\C{\mathbb{C}}
\def\A{\mathbb{A}}

\def\bbF{{\mathbb{F}}}
\def\F{{\bm{F}}}
\def\f{{\bm{f}}}
\def\g{{\bm{g}}}


\def\balpha{\bm{\alpha}}

\def\jac{{\mathrm{jac}}}
\newcommand{\se}{\mathrm{SE}(3)}
\newcommand{\confg}{\mathcal{C}}
\newcommand{\pl}{\mathrm{Pl}}
\newcommand{\bs}{\backslash}

\def\alert#1{{\textcolor{red}{#1}}}

\def\hlight#1{\textcolor{Blue}{{#1}}}
\def\alert#1{\textcolor{red}{({\bf todo: #1})}}
\def\anrinput#1{}%{\textcolor{red}{{#1}}}
%\def\anrinput#1{\textcolor{red}{{#1}}}
\def\paris#1{\textcolor{Olive}{#1}}
\def\ls2n#1{\textcolor{DarkGreen}{#1}}
\def\jku#1{\textcolor{magenta}{#1}}
%\def\irobot#1{\textcolor{Violet}{#1}}
\def\mohab#1{\textcolor{Brown}{#1}}

\def\ACRONYM{\color{darkgoldenrod}{\texttt{msolve} -- A library for solving multivariate polynomial systems}}
\def\msolve{{\textsc{msolve}}\xspace}


% \newtheorem{pbm}{Problem}
% \newtheorem{definition}{Definition}
% \newtheorem{theorem}[definition]{Theorem}
% \newtheorem{corollary}[definition]{Corollary}
% \newtheorem{proposition}[definition]{Proposition}
% \newtheorem{lemma}[definition]{Lemma}
% \newtheorem{remark}[definition]{Remark}


\titlehead{\ACRONYM}
\title{How to solve multivariate polynomial systems with {\msolve}?}
\date{}

\usepackage[
%  citestyle=authoryear,
  bibstyle=authoryear,
  citestyle=numeric,
  bibstyle=numeric,
%  dashed=false,
  backend=biber,
  isbn=false,
  doi=false,url=false,
  maxcitenames=50,
  hyperref=true,
  maxbibnames=20,
  firstinits=true
  ]{biblatex}
  \bibliography{biblio}

\begin{document}

\maketitle

\tableofcontents

\section{Introduction}

\msolve~is a C library for solving multivariate polynomial systems of equations. 
It relies on computer algebra, a.k.a.\ symbolic computation, algorithms to
compute \emph{algebraic} representations of the solution set from which
many, if not all, informations can be extracted.

Solving polynomial systems with \msolve~is \emph{global} by contrast to
\emph{local} numerical routines. The use of computer algebra methods allow also
the user to bypass classical numerical issues encountered by numerical methods
for polynomial system solving.

\msolve~relies mainly on Gr\"obner bases algorithms (see below for a some basic
definitions and properties). It is highly optimized, uses \texttt{AVX2}
vectorization instructions and multi-threading.

It uses the \texttt{GMP} library (handling multi-precision integers) and the
\texttt{FLINT} library (handling arithmetics of univariate polynomials).

\msolve~can be downloaded from
\begin{center}
  \url{https://msolve.lip6.fr}
\end{center}
where binaries (for \texttt{x86} processors runing Linux operating systems) and
the source files are provided.

\msolve~is designed for $64$ bit architectures, with \texttt{AVX2} instructions. 

\section{Preliminaries}

Polynomial systems arise in a wide range of applications in engineering and
computing sciences. What it means to ``solve'' a polynomial system mainly
depends on the application, the \emph{domain} where the coefficients lie and the
properties of the solution set.

For instance, in many applications of engineering sciences, the coefficients lie
in the \emph{field} of rational numbers $\Q$. Sometimes, these coefficients
appear as \emph{floating point} numbers but since floating point arithmetics is
not fully convenient to handle reliably non-linear systems, the end-user may
coerce these coefficients into rational numbers. There, using continued
fractions is quite useful as it allows one to obtain reliable approximations
with integers of small \emph{height} (hence small length when represented with
bits). 

All in all, one ends up with polynomial equations in the polynomial ring
$\Q[x_1, \ldots, x_n]$, i.e. the set of polynomials with coefficients in $\Q$
and involving indeterminates $x_1, \ldots, x_n$.

In such applications, the end-user will most of the time expect informations on
the solutions with coordinates in the field of real numbers $\R$ (for instance
in robotics or biology) or in the field of complex numbers $\C$ (for instance in
signal theory).

When the number of solutions of such systems in $\C^n$ is finite (one says that
such systems have dimension at most zero), then it is well-known that the
coordinates of the solutions can be parametrized by an \emph{algebraic number}
at which one evaluates a rational fraction. In other words, there exist
univariate polynomials $(w, v_1, \ldots, v_n)$ in $\Q[t]$ such that $w$ is
square-free and the solution set in $\C^n$ can be written as
\[
  \left \{
  \left (\frac{v_1(\vartheta)}{w'(\vartheta)}, \ldots,
  \frac{v_n(\vartheta)}{w'(\vartheta)}\right ) \quad \st \quad w(\vartheta) = 0
\right \}
\]
where
$w'=\frac{\ud w}{\ud t}$.
% $w'=\frac{\partial w}{\partial t}$.
Note that, with such an encoding
(which is called a rational parametrization of the solution set), one can
approximate at arbitrary precision the solutions of the input system (by isolating
the ones of the univariate polynomial $w$).
Note also that, since $w$ is square-free, $w$ and $w'$ are co-prime and the
denominator in the above parametrization can be replaced by a constant.

Remark that the total number of solutions in $\C^n$ equals the degree of $w$.
Hence, when the input system has no complex solution, $w$ is a non-zero
constant. Also, the number of real roots to the input system is the number of
real roots of $w$.

Note also that this encoding loses the number of solutions counted with
\emph{multiplicities} ; this number is called the \emph{degree} of the input
system (which always dominates the number of complex solutions). 

For systems in $\Q[x_1, \ldots, x_n]$, \msolve~is able to decide if they have
finitely many complex solutions and, in that case, returns a rational
parametrization of the solution set in $\C^n$. It also counts and isolates the
real roots of the polynomial $w$. We refer to Section~\ref{sec:zerodim} for
details on how \msolve~encodes such outputs.

When the input system has infinitely many complex solutions, \msolve~can output
a \emph{Gr\"obner basis} of the \emph{ideal} generated by all input polynomials
; this is a family of polynomials which is obtained by taking appropriate
algebraic combinations of the input polynomials and from which one can extract
several properties of the solution set, in particular its \emph{dimension} and
its \emph{degree} (see Section~\ref{sec:posdim}). 

For applications in cryptology and/or coding theory, polynomial systems lie in
some polynomial ring $\bF[x_1, \ldots, x_n]$ where $\bF$ is some finite field. In
that situation, \msolve~only supports prime field $\frac{\Z}{p\Z}$ with $p <
2^{31}$. Denoting by $\overline{\bF}$ an algebraic closure of $\bF$, such systems
have dimension at most $0$ when they have finitely many solutions in
$\overline{\bF}^n$, else they have positive dimension. For systems of dimension
at most zero, a rational parametrization as above can be computed when the
characteristic of the field is large enough. 

\section{Input file format}

Consider the following polynomial system of equations
\begin{center}
$
\begin{array}{rcl}
x+2y+2z-1 &= &0\\
x^2+2y^2+2z^2-x &= &0\\
2xy+2yz-y &= &0
\end{array}
$
\end{center}
in $\Q[x,y,z]$.

In order to solve it with \msolve~one simply produces a file with the following content
\begin{tcolorbox}
\begin{verbatim}
x,y,z
0
x+2*y+2*z-1,
x^2+2*y^2+2*z^2-x,
2*x*y+2*y*z-y
\end{verbatim}  
\end{tcolorbox}
Hence the structure of input files to \msolve is as follows:
\begin{enumerate}
\item the first line contains the variables of the input system, separated with
  a comma ; 
\item the second line contains the characteristic of the field over which
  computations are performed ; 
\item the next lines contain polynomials, in expanded form, separated by a comma
  (except the last one) and with a line break. 
\end{enumerate}



When one wants to solve this system over $\frac{\Z}{65521\Z}$ one just replaces
$0$ by $65521$ in the second line. Note that in the positive characteristic case
the coefficients used should be smaller or equal to $2^{31}-1$.
\section{Solving polynomial systems of dimension at most zero}\label{sec:zerodim}

Let $\K$ be a field and $\overline{\K}$ be an algebraic closure of $\K$. We
consider polynomials $(f_1, \ldots, f_s)$  in $\K[x_1, \ldots,
x_n]$ and $V$ be the set of solutions to the system
\[
f_1=\cdots=f_s=0
\]
in $\overline{\K}^n$.

When $V$ is
empty,
% in $\overline{\K}[x_1, \ldots, x_n]$,
the
output of \msolve~is
\begin{verbatim}
[1, []]
\end{verbatim}

When $V$ is finite, say $V$ is the union of the $\ell$ points $\balpha_1,
\ldots, \balpha_\ell$ in $\overline{\K}^n$ with $\balpha_i = (\alpha_{i,1},
\ldots, \alpha_{i,n})$,
% When the input system has finitely many solutions in
% $\overline{\K}^n$,
the
output is encoded as follows:
% \begin{verbatim}
% [dim, [deg, nbsols, form, vars, [lw, lwp, param]]]
% \end{verbatim}
% where
% \begin{itemize}
% \item \texttt{dim} is $0$ (which indicates that the input system has finitely
%   many solutions);
% \item \texttt{deg} is the number of solutions counted with multiplicities in
%   $\overline{\K}^n$;
% \item \texttt{nbsols} is the number of solutions in $\overline{\K}^n$; 
% \item \texttt{form} is the linear form $t$;
% \item \texttt{vars} is the list of the variables which are parametrized;
% \item \texttt{lw} is the encoding of the eliminating polynomial $w$;
% \item \texttt{lwp} is the encoding of the denominator used in the rational
%   parametrization;
% \item \texttt{param} is the list of the output parametrizations as described
%   above ; the $i$th one corresponds to the $i$th element of \texttt{vars}.
%   % \textcolor{magenta}{I am not sure this is still the case, maybe I used a reverse
%   %   ordering.} 
% \end{itemize}
\begin{verbatim}
[dim, nbvar, deg, varstr, linform, lw, lwp, param]
\end{verbatim}
in characteristic $0$, where
\begin{itemize}
\item \texttt{dim} is $0$ (which indicates that the input system has finitely
  many solutions);
\item \texttt{nbvar} is the number of variables;
\item \texttt{deg} is the number of solutions counted with multiplicities in
  $\overline{\K}^n$;
% \item \texttt{nbsols} is the number of solutions in $\overline{\K}^n$; 
\item \texttt{varstr} is the list of the variables which are parametrized;
\item \texttt{linform} is the linear form $t$, it is empty if there is none;
\item \texttt{lw} is the encoding of the eliminating polynomial $w$
  either in $t$ or the last variable of \texttt{varstr};
\item \texttt{lwp} is the encoding of the denominator used in the rational
  parametrization;
\item \texttt{param} is the list of the output parametrizations as described
  above ; the $i$th one corresponds to the $i$th element of \texttt{varstr}.
  % \textcolor{magenta}{I am not sure this is still the case, maybe I used a reverse
  %   ordering.} 
\end{itemize}
\begin{verbatim}
[char, nbvar, elim, den, param]
\end{verbatim}
in positive characteristic, where
\begin{itemize}
\item \texttt{char} is the characteristic of the field;
% \item \texttt{dim} is $0$ (which indicates that the input system has finitely
%   many solutions);
\item \texttt{nbvar} is the number of variables;
\item \texttt{elim} is the encoding of the eliminating polynomial of
  the last variable;
\item \texttt{den} is the encoding of the denominator used in the rational
  parametrization, it is always $1$ at the moment;
\item \texttt{param} is the list of the output parametrizations as described
  above ; the $i$th one corresponds to the $i$th element of \texttt{vars}.
  % \textcolor{magenta}{I am not sure this is still the case, maybe I used a reverse
  %   ordering.} 
\end{itemize}


%\textcolor{magenta}{Here we should give (small) examples.}
\section{Solving polynomial systems of positive dimension}\label{sec:posdim}

\section{Julia interface to \msolve}
The Julia interface to \msolve is part of the official Julia package
\texttt{GroebnerBasis.jl}. You can install the package via the following
commands inside a Julia session:\\[1em]
\texttt{using Pkg\\
Pkg.add(“GroebnerBasis”)}\\[1em]
Once the package is loaded via \texttt{using GroebnerBasis} one can call the function
\texttt{msolve()} which returns, if any, the solutions of the given input
system of multivariate polynomials. The most common calling convention is as follows:\\[1em]
\texttt{solutions = msolve(I)}.\\[1em]
where
\begin{itemize}
    \item \texttt{I} is of type \texttt{Singular.sideal} and
    \item \texttt{solutions} is of type \texttt{Array\{Array\{Rational\{Int64\},1\},1\}}.
\end{itemize}
The most common options for calling \texttt{msolve()} in Julia are:
\begin{itemize}
    \item \texttt{info\_level} with values \texttt{0} (no information printing;
        default), \texttt{1} (slight information printing on comptutational
        status) or \texttt{2} (full information printing also on intermediate
        steps),
    \item \texttt{la\_option} for the linear algebra variant to be chosen inside
        F4: \texttt{2} for exact linear algebra and tracing multi modular
        computations (default) or \texttt{44} for probabilistic linear algebra
        with independent modular computations;
    \item \texttt{precision} for the bit precision with which the solutions are
        computed from the rational parametrization. Default is \texttt{64}.
    \item \texttt{get\_param} for returning, besides the solutions, also the
        rational parametrization for the solution set. Default is
        \texttt{false}.
\end{itemize}
So using \texttt{msolve()} with probabilistic linear algebra, the most verbose
information printout and a precision of $80$ one would call\\[1em]
\texttt{solutions = msolve(I, la\_option=44, info\_level=2,
precision=80)}.\\[1em]
Returning also the rational parametrization of the solution set one would
call\\[1em]
\texttt{param, solutions = msolve(I, get\_param=true)}.\\[1em]
Having the rational parametrization \texttt{param} one can further solve trying,
for example, higher precision:\\[1em]
\texttt{newly\_computed\_solutions = solve\_rational\_parametrization(param)}\\[1em]
You can get a full list of options for \texttt{msolve()} in Julia by typing
inside Julia\\[1em]
\texttt{? msolve()}
\section{Maple interface to \msolve}
The Maple interface to \msolve can be found on the \msolve homepage or in the
\msolve binary package.
Having loaded the interface one can call the function
\texttt{MSolveRealRoots()} in the following way:\\[1em]
\texttt{results = MSolveRealRoots(F, vars, msolve\_binary, in\_file, out\_file, get\_parametrization)}\\[1em]
    where \texttt{F} denotes a polynomial system in variables \texttt{vars},
    \texttt{msolve\_binary} gives the path to the binary (default is
    \texttt{“../binary/msolve”}), 
\texttt{in\_file} is the name of the input file for \msolve (default is
\texttt{“/tmp/in.ms”}), 
\texttt{out\_file} is the name of the output file for \msolve (default is
\texttt{“/tmp/out.ms”}), and 
\texttt{get\_parametrization} defines if the rational parametrization of the
solution set is also returned (default is
\texttt{0}, i.e. the parametrization is not returned). 
 Depending on parameter \texttt{get\_parametrization}
\texttt{results} consists only of the solutions or it is an array consisting of
the rational parametrization as first entry and the solutions as second entry.

\section{Sage interface to \msolve}
The Sage interface to \msolve can be found on the \msolve homepage or in the
\msolve binary package. Starting
Sage we load the interface:\\[1em]
\texttt{load(“msolve-to-sage-file-interface.sage”)}.\\[1em]
Defining a multivariate polynomial ring \texttt{R} and a list \texttt{F} of
polynomials in \texttt{R} one can solve the multivariate polynomial system
defined by \texttt{F} via calling\\[1em]
\texttt{results = MSolveRealRoots(}\\
\texttt{F, mspath=“/path/to/msolve/binary”,
v=verbosity, p=get\_parametrization)}\\[1em]
Parameter \texttt{mspath} states the path to the msolve binary (default is
\texttt{“../binary/msolve”}).
The verbosity level can be set to \texttt{0} (no information printing; default),
\texttt{1} (slight information printing on computational status) or \texttt{2}
(full information printing also on intermediate steps).
The \texttt{p} parameter can be set to \texttt{0} (rational parametrization of
solution set is not returned) or \texttt{1} (returns also the rational
parametrization of the solution set). Depending on parameter \texttt{p}
\texttt{results} consists only of the solutions or it is an array consisting of
the rational parametrization as first entry and the solutions as second entry.
\section{Credits}

The main developers of \msolve~are J\'er\'emy Berthomieu, Christian Eder and
Mohab Safey El Din. It relies on original implementations of Faug\`ere's $F_4$
algorithm \cite{F4} as well as Faug\`ere and Mou's Sparse-FGLM algorithm
\cite{SparseFGLM}. We are grateful to Huu Phuoc Le and Jorge Garcia Fontan for a
preliminary version of the Maple interface as well as Rémi Prébet for a
preliminary version of the Sage interface.  

\renewcommand*{\bibfont}{\small}
  \printbibliography
\end{document}

%%% Local Variables:
%%% mode: latex
%%% TeX-master: t
%%% End:
